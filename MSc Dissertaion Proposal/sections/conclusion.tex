\chapter{Conclusions}

%\lipsum  % Replace with your text
By writing this project background report, the author finished the literature survey, project analysis, risk analysis, and work planning.

In the literature survey part, the author reviewed the classic theories on prosody such as the work of Julia Hirschberg and Janet Pierrehumbert, as well as the emerging researches on voice quality and speech style. The author has planned to select pitch accent, prosodic phrasing and tune as main features and use ToBI model to quantitatively analyze the spoken utterances in Switchboard corpus. The prosodic feature extraction software to use will probably be Praat. 

As for language modelling, metrics such as TF-IDF, perplexity and RIT annotation are supposed to be adopted in order to implement a statistical word importance prediction model. Additionally, neural network based approach should not be overlooked such as word-embedding tool BERT and the Long-Short Term Memory model.

The objectives and research questions of the project have been examined in detail.  The author also analysed each sub-task in the implementation stage and listed the preferred methods. A risk register is provided which lists 10 hazards and actions to avoid or decrease the impact.

This project is divided into 3 stages: theoretical framework, experiment implementation and project dissertation. A  Gantt chart is draw to show the timeline, sub-tasks, milestones and dependencies between tasks. For a good project organisation, stage retrospectives and an experiment journal will also be written.