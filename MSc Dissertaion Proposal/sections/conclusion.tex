\chapter{Conclusions}

%\lipsum  % Replace with your text

In speech communication, prosodic patternings (pitch, duration and energy) are utilised to draw the listener's attention to the most informative words, and the informativeness is closely related to the predictability of the spoken word. It is hypothesised that the prominent prosodic features correlate to word predictability. This experimental research tested the hypothesis by computing the Spearman's correlation coefficients of the variable pairs -- each consisting of an acoustic prosodic feature and a numeric representation of word predictability. We used TF-IDF value, raw word probability and a set of human-annotated word importance scores as the measurements of word predictability.

We have shown that the selected word-level, non-speaker-nomorlised, prominent prosodic features have weak to medium correlation with word prodictability. Ranked by the strength of correlation, the prosodic features follow such a order: duration - excursion size - maxf0 location - maxf0 velocity - mean intensity; from the aspect of word predictability measurements, the correlation ranks from high to low is: raw word probability - word importance scores - TF-IDF.

To test the effectiveness of the selected prosodic features in predicting important words given the proved correlation, we have implemented a simple-structured neural word importance classifier trained for a 3-class classification task. It achieves an accuracy score of 62.54, which is very close to Kafle's model with a much complicated feature settings, and is comparitable to the text-based models. Concequently, this exhibits the usefulness of our prosodic features and shows the potential of the speech-based models in real life prediction tasks given the impracticalness to attain text transcripts.

Scientific understanding can be only as accurate as the level of detail we choose in our observations. For further research on this topic, one direction could go to syllable-level feature extraction as literature proves that prosodic prominence is more detectable with sub-word features such as syllable energy and syllable duration. Moreover, speaker normalisation could be applied to reduce the affect of sex, articulation variation and microphone condition. Furthermore, the ingratation of lexical and acoustic prosodic cues is also a promising direction to a more robust word importance detector.

