\chapter{Introduction}
% replace with your text

%
%Not all words are equally important to the meaning
%of a spoken message.
%
%
%Identifying the importance
%of words is useful for a variety of tasks includ-ing text classification and summarization %(Hong
%and Nenkova, 2014; Yih et al., 2007).
%
%While spoken messages include prosodic cues
%that focus a listener’s attention on the most im- portant parts of the message (Frazier et %al., 2006), such information may be omitted from a text tran- script
%
%
%Our goal is to explore the versatility of
%speech-based (text-independent) features for word
%importance modeling.

\section{Background}
 One of the key functions of the "prosodic" patterning in speech is thought to be to draw attention to the key information-bearing elements of a spoken utterance. In other words, the most important parts of an utterance are often associated with prosodic "prominence" such as increased loudness, large changes in the intonation, and extended segment duration. From a linguistic perspective, the key information-bearing elements are those which are the least predictable. Therefore, taking these two aspects together, it can be hypothesised that there should be a correlation between the predictability of words (known as "perplexity") and the major prosodic prominence in an utterance.

\section{Aims and Objectives}
% replace with your text

The aim of this experimental project is to test this hypothesis. Hence the project will involve the use of (i) appropriate speech processing algorithms to derive prosodic features from a selected corpus of recorded utterances, (ii) appropriate text processing algorithms to compute the predictability of each individual word in the corresponding transcripts, and (iii) statistical tools to analyse the putative relationship between the two.

%\subsection{constraints}

%\lipsum % replace with your text

\section{Outline}
This report consists of 5 chapters. Chapter 1 is an introduction to the project. The main part of the report is Chapter 2 Literature Survey in which the previous work on prosody and word importance have been reviewed. It also introduces the database and software to be used. In Chapter 3, the author analyzed the research questions, implementation options and the ethical issues in this project. Chapter 4 provides a risk register and a Gantt style work plan. Chapter 5 is the conclusion.    

%\lipsum % replace with your text


%\section{Overview of the Report}
%
%\lipsum % replace with your text
%
%\subsection{A subsection}
%
%\lipsum % replace with your text
%
%\subsection{Another subsection}
%
%\lipsum % replace with your text