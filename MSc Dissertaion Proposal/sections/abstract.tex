\chapter*{\Large \center Abstract}

% Guidance of how to write an abstract/summary provided by Nature: https://cbs.umn.edu/sites/cbs.umn.edu/files/public/downloads/Annotated_Nature_abstract.pdf

In conversational speech, prosodic pattenings such as pitch, duration and energy aid the listerner in caputring the most informative words in the message.
This experimental project aims to test a hypothesis that the prominent prosodic features in human speech correlate with the predictability of the word being spoken. We investigated Spearman's correlaiton between selected acoustic features and three numeric measurements of word predicatability--TF-IDF, raw word probability and word importance scores--using SwitchBoard corpus as the dataset. It has been proved that the selected word-level, non-speaker-nomorlised, prosodic features have weak to medium correlation with word prodictability. In addition, an neural archtecture has been implemented for a 3-class word importance prediction task with the selected prosodic features as the input. Our model yields comparable performance with state-of-the-art models and demostrates the effectiveness of our prosodic features and the potential of speech-based models.

key words: prosody, prosodic prominence, word predictability, word importance