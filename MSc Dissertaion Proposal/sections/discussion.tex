\chapter{Results and Discussion}

Table () summarise the correlations between prosodic features, word importance and word probability, any two of them as a pair. Significant correlation figures are coloured and the colour changes with the significance. 

\section{Correlation: Prosodic Features - Word Importance }

In general, it could be concluded that the prominent prosodic features do correlate to the word importance, but the correlation is not very significant. The prominent prosodic features are extended duration, largely changed pitch and energy. According to table (), duration is positively correlated to word importance with a Spearman correlation coefficient valued 0.471, the greatest among others; excursion size correlation value, 0.251,  is comparatively great too, and the mean f0 follows with a value of 0.121. As for the raised voice, the correlation value for mean intensity is also commendable. The max f0 location feature is related to both duration and pitch, and the correlation, valued 0.231, is relatively strong too.

table(Prosodic Features - Word Importance Correlation)

Have we missed something? Yes, the first derivative of f0--max velocity and final velocity. They seem not very correlated. However, the derivative value can be positive or negative. It is the absolute value that matters. When using the absolute value to compute the correlation, it could be observed that the correlation value between max velocity and the word importance increased from 0.074 to a commendable level at 0.141 whereas the coefficient for final velocity didn't change prominently.

table(Velocity - Word Importance Correlation)

\section{Word Importance - Word Predictability Correlation}
As it is clearly shown in table(), given a token size of 25,000, word importance has a strong negetive correlation with word predictability with the Spearman correlation coefficient valued -0.700. It proves our hypothesis that the more importance the word in a sentence, the less predictable it is. 

table(Word Importance - Word Predictability Correlation)

\section{Correlation: Prosodic Features - Word Probability}
Experiments proved that there is correlation between prominent prosodic features and word predictability as it is shown in table(), and the correlation is negative. With one prominent prosodic feature, for instance mean intensity, increasing, the word predictability tends to decrease, hence the word is less predictable. With a token size of 25,000, the most correlated feature is duration with its correlation coefficient valued -0.565. Excursion size and max f0 location rank second with a value around -0.28. The figure for mean intensity, -0.116, is noticeable, too. The p-value (section ) for each aforementioned prosodic features is much smaller than the significance level 0.05, indicating that the correlation is statistically highly significant.

table(Prosodic Features - Word Predictability Correlation, 25,000)
table(Prosodic Features - Word Predictability Correlation, 2m)
table(Prosodic Features - Word Predictability Correlation, 4m)

\section{Correlation: Prosodic Features - TF-IDF}
As for the relationship between the selected prosodic features and TF-IDF value of a word, table () reflects that only duration, excursion size and max f0 location seem to have weak correlation with TF-IDF.

table()
 \section{Performance and Analysis: MLP-Softmax Model }
  table(Results)
 
After 100 epochs of training, the MLP-Softmax model acchieved a accuracy score of 62.54. To evalutate the result, we compared it to the performances of another 4 models with the similar experiment setup(site ). (cite, 2019) also implemented a accoustic-prosodic based word importance classifier whereas the speech features they used were more complicated, in total 30 features related to pitch(10), energy (11), voicing (3) and spoken-lexical elements (6). It was a sequence-to-sequence model using LSTM to capture the context information.

As it is described in (cite 2019), althought the text-based models outperformed the speech-based models because of the significant semantic information available, their LSTM-Softmax model were competitive with an accuracy score of 63.72 especially in performance with text-based models noised by human-made word errors. According to (cite, Barker) it is typical for modern ASR systems to have a WER aound 30\% in many real-world scenarios. Thus, speech-based systems have huge potential.

Compared with the LSTM-Softmax model, our simple MLP-Softmax model only have 1/3 of the parameter numbers, yet has acchieved a very similar performance without using any context information. It could be inferred that, considering only the speech signals, the importance of a word is more related to the word itself. Additionally, the prominent prosodic features such as duration, intensity, largely changed pitches seem to be more important features to be selected.






